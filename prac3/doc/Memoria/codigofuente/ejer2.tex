%%%%%%%%%%%%%%%%%%%%%% Optimización del rendimiento de un procesador CMP - EJERCICIO 2 %%%%%%%%%%%%%%%%%%%%%%
\section{Ejercicio 2}
\subsection{Enunciado}
\begin{ejer}
    \textbf{2.- Paralelización.} Optimiza el rendimiento de este micro-kernel usando \texttt{OpenMP} para paralelizar su
    ejecución. Ten en cuenta que pueden aparecer condiciones de carrera. Evalúa la aceleración que
    has obtenido en función del número de núcleos que utilizas y comenta los resultados obtenidos.
\end{ejer}
\subsection{Desarrollo}
\begin{listing}[firstnumber=16]
    @@ -17,4 +17,4 @@
    //PARALELIZACION
    #define DEFAULT_SIZE 32
    void BinParticles(const InputDataType & inputData, BinsType & outputBins) {
        static int *iX = (int*)_mm_malloc(inputData.numDataPoints*sizeof(int), DEFAULT_SIZE); 
        static int *iY = (int*)_mm_malloc(inputData.numDataPoints*sizeof(int), DEFAULT_SIZE);
    #pragma omp parallel for reduction(+:outputBins)
        for (int i = 0; i < inputData.numDataPoints; i++) {
            const FTYPE x = inputData.r[i]*COS(inputData.phi[i]); 
            iX[i] = int((x - xMin)*binsPerUnitX);
            const FTYPE y = inputData.r[i]*SIN(inputData.phi[i]); 
            iY[i] = int((y - yMin)*binsPerUnitY); 
            ++outputBins[iX[i]][iY[i]];
        } 
    }
\end{listing}
\par Obtengo el rendimiento usando \texttt{OpenMP} para paralelizar su ejecución.
\begin{listing}[style=consola]
    Baseline performance: 8.35e-02 GP/s
    
    Benchmarking...
    
    Trial    Time, s Speedup    GP/s *
        1  4.422e-01    1.82 1.52e-01 **
        2  3.260e-01    2.46 2.06e-01 **
        3  3.330e-01    2.41 2.02e-01
        4  3.435e-01    2.34 1.95e-01
        5  3.305e-01    2.43 2.03e-01
        6  3.269e-01    2.46 2.05e-01
        7  3.381e-01    2.38 1.99e-01
        8  3.268e-01    2.46 2.05e-01
        9  3.347e-01    2.40 2.01e-01
        10 3.498e-01    2.30 1.92e-01
    --------------------------------------------------------- 
    Optimized performance: 2.40 2.00e-01 +- 4.46e-03 GP/s
    ---------------------------------------------------------
\end{listing}
\par La paralelización es la técnica que ha obtenido mayor rendimiento al utilizarse aisladamente, 
el compilador de gcc permite el parámetro outputBins en reduction que es una matriz, en el compilador 
de intel se hace manualmente.
