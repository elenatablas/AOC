%%%%%%%%%%%%%%%%%%%%%% Optimización del rendimiento de un procesador CMP - EJERCICIO 3 %%%%%%%%%%%%%%%%%%%%%%
\section{Ejercicio 3}
\subsection{Enunciado}
\begin{ejer}
    \textbf{3.- Vectorización.} Optimiza el rendimiento de este micro-kernel usando el compilador para vectorizar
    automáticamente su ejecución. Observa el informe del compilador para encontrar los problemas
    que pueden aparecer para realizar dicha vectorización y estudia si es adecuado cambiar de los
    datos de \texttt{AoS} a \texttt{SoA}. Evalúa la aceleración que has obtenido en función del tamaño de la unidad
    vectorial y del número de unidades vectoriales que tengas, y comenta los resultados obtenidos.
\end{ejer}
\subsection{Desarrollo}
\begin{listing}[firstnumber=16]
    @@ -17,4 +17,4 @@
    // VECTORIZADO
    #define DEFAULT_SIZE 32
    void BinParticles(const InputDataType & inputData, BinsType & outputBins) {
        static int *iX = (int*)__builtin_assume_aligned((int*)_mm_malloc(inputData.numDataPoints*sizeof(int), DEFAULT_SIZE), DEFAULT_SIZE);
        static int *iY = (int*)__builtin_assume_aligned((int*)_mm_malloc(inputData.numDataPoints*sizeof(int), DEFAULT_SIZE), DEFAULT_SIZE);
    #pragma omp simd
        for (int i = 0; i < inputData.numDataPoints; i++) { 
            const FTYPE x = inputData.r[i]*COS(inputData.phi[i]); 
            iX[i] = int((x - xMin)*binsPerUnitX);
            const FTYPE y = inputData.r[i]*SIN(inputData.phi[i]); 
            iY[i] = int((y - yMin)*binsPerUnitY); 
            ++outputBins[iX[i]][iY[i]];
        } 
    }
\end{listing}
\par Obtengo el rendimiento usando el compilador para vectorizar automáticamente su ejecución.
\begin{listing}[style=consola]
    Baseline performance: 7.85e-02 GP/s 
    
    Benchmarking...
        Trial Time, s       Speedup     GP/s *
            1 1.089e+00     0.79    6.16e-02 **
            2 9.023e-01     0.95    7.44e-02 **
            3 8.709e-01     0.98    7.71e-02
            4 8.934e-01     0.96    7.51e-02
            5 8.742e-01     0.98    7.68e-02
            6 8.668e-01     0.99    7.74e-02
            7 8.566e-01     1.00    7.83e-02
            8 8.435e-01     1.01    7.96e-02
            9 8.340e-01     1.03    8.05e-02
            10 8.512e-01    1.00    7.88e-02
    ---------------------------------------------------------
    Optimized performance: 0.99 7.79e-02 +- 1.60e-03 GP/s
    ---------------------------------------------------------
\end{listing}
\par La vectorización, que consiste en indicar que los datos están alineados en la caché no mejora 
el código. Incluso podría decir que lo ha empeorado. Los pragmas de gcc nos distintos a los de icc, 
por tanto, he tenido que buscar información de como se hacía por internet, en la bibliografía están 
las referencias. El informe generado es ilegible en gcc, aunque tanto en icc como en gcc indican 
que no se el compilador no sabía vectorizarlo directamente.
