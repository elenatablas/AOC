%%%%%%%%%%%% INTRODUCCIÓN  %%%%%%%%%%%%

\begin{center}
	{\fboxrule=4pt \fbox{\fboxrule=1pt
		\fbox{\LARGE{\bfseries Introducción}}}} \\
	\addcontentsline{toc}{chapter}{Introducción}
	\rule{15cm}{0pt} \\
\end{center}

\pagenumbering{roman}
 
 \lettrine[lines=3, depth = 0]{E}{n} este documento este documento explicaremos cómo hemos llevado a 
 cabo el proyecto de prácticas para la asignatura 
 Ampliación de Sistemas Operativos que consiste en realizar los ejercicios de xv6 siguiendo las 
 especificaciones indicadas en los boletines de prácticas.
 
 \par La práctica ha sido desarollada en un entorno \texttt{Linux} usando el emulador \texttt{EMU}. El lenguaje de
 programación utilizado ha sido \texttt{C}. 

 \par También, hemos utilizado \texttt{Git} como gestor de versiones, que nos ha permitido
 trabajar de forma paralela y obtener una versión incremental del proyecto.

 \par Esta memoria ha sido creada en el sistema de composición de textos \LaTeX.


\newpage
\pagenumbering{arabic}