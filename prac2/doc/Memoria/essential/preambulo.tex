
%Margenes
\usepackage[margin=1 in, includefoot]{geometry}

% Acentos, letras, etc
\usepackage[T1]{fontenc}
\usepackage[utf8]{inputenc}
\usepackage[spanish]{babel}
\addto\captionsspanish{
\renewcommand{\chaptername}{}
}
\selectlanguage{spanish}
\usepackage{lettrine, Zallman}
\renewcommand\LettrineFontHook{\Zallmanfamily}

\usepackage{times}
\usepackage{ulem} % texto tachado
\usepackage{soul}
\usepackage{lipsum}
\usepackage{comment}

% Header and Footer Stuff
\usepackage{fancyhdr}
\pagestyle{fancy}
\fancyhead[L,RO]{}
\fancyhead[LO,R]{}
\lhead[\leftmark]{Pérez González-Tablas, Elena}
\rhead[Nombre Autor]{AOC, 2021/22}
\renewcommand{\footrulewidth}{1pt}

% Licencia
\usepackage[
    type={CC},
    modifier={by-nc-sa},
    version={3.0},
]{doclicense}

% Enumerados
\renewcommand{\labelenumii}{\alph{enumii}$)$ }

%hipervínculos, referencias, citas, etc
\usepackage[colorlinks=true, 
		    linkcolor=blue, 
		    citecolor = black,
		    urlcolor = blue]{hyperref}

% Símbolos matemáticos
\usepackage{amsmath}

% Figuras
\usepackage{float}
\usepackage{subfigure}
\usepackage{graphicx}
\usepackage{wrapfig}

%https://ondahostil.wordpress.com/2017/05/17/lo-que-he-aprendido-cuadros-de-texto-de-colores-en-latex/
\usepackage{lmodern}
\usepackage{tcolorbox}
\tcbuselibrary{listingsutf8}
\newtcolorbox[auto counter,number within=subsection]{ejer}[1][]
{colback=blue!5!white,colframe=blue!75!black,fonttitle=\bfseries, title=#1}

\usepackage{color} % color en texto
%Creación de mis propios colores

\definecolor{rojoOscuro}{RGB}{153,0,0}
\definecolor{verdeOscuro}{RGB}{0,153,0}
\definecolor{gray97}{gray}{.97}
\definecolor{gray75}{gray}{.75}
\definecolor{gray45}{gray}{.45}

% Código
\usepackage{listings}
% Letra para código http://www.rafalinux.com/?p=599
\usepackage{inconsolata}

\lstset{ 
	language=C,
%
	frame=Ltb,
	framerule=0pt,
	aboveskip=0.2cm,
	framextopmargin=3pt,
	framexbottommargin=3pt,
	framexleftmargin=1.0pt,
	framesep=0.2pt,
	rulesep=1.0pt,
	backgroundcolor=\color{gray97},
	rulesepcolor=\color{black},
%
	stringstyle=\ttfamily,
	showstringspaces = false,
	basicstyle=\small\ttfamily,
	commentstyle=\color{gray45},
	keywordstyle=\bfseries,
%
	numbers=left,
	numbersep=7pt,
	numberstyle=\tiny,
	numberfirstline = false,
	breaklines=true,
%
	morecomment=[f][\color{gray45}]{@@},
  	morecomment=[f][\color{verdeOscuro}]{+\ },
  	morecomment=[f][\color{rojoOscuro}]{-\ },
}

% minimizar fragmentado de listados
\lstnewenvironment{listing}[1][]
{\lstset{#1}\pagebreak[0]}{\pagebreak[0]}

\lstdefinestyle{consola}
{basicstyle=\scriptsize\bf\ttfamily,
backgroundcolor=\color{gray75},
}

\graphicspath{ {images/} }

\usepackage{array,tikz}
\usetikzlibrary{matrix,positioning}

\tikzset{  % Define some styles
  my table/.style = {
    matrix of nodes,
    nodes = {minimum width=3em, minimum height=2.5em, inner sep=3pt},
    draw,
    inner sep=0pt,
  },
  my arrow/.style = {
    -latex,
    shorten >= -5pt,
    shorten <= -5pt,
    black!50!black!60,
  }
}
% Macro to draw the lines between rows
\def\drawLineBelowRow#1#2{% #1: row number, #2: matrix name
  \draw (#2.west|-#2-#1-1.south) -- (#2.east|-#2-#1-1.south);
}

