  %%%%%%%%%%%%% COMENTARIOS SOBRE LA PRÁCTICA %%%%%%%%%%%%%%%%
  \begin{center}
	{\fboxrule=4pt \fbox{\fboxrule=1pt
		\fbox{\LARGE{\bfseries Comentarios sobre la práctica}}}} \\
	\addcontentsline{toc}{chapter}{Comentarios sobre la práctica}
	\rule{15cm}{0pt} \\
\end{center}

\par No me ha gustado esta práctica porque tengo la sensación de que he aprendido bastante pero no lo suficiente como para saber
resolver esta tarea correctamente. Los videos de las clases han servido de mucha ayuda para intentar resolverla y los boletines.
Aunque en ellos solo se planteaban algunos casos específicos y al intentar abordar otros diferentes como los de esta tarea me ha
costado bastante. En la vectorización era fácil comprender el informe pero el código en ensamblador ha sido costoso interpretarlo
y lo necesitaba para saber como había el compilador implementado las mejoras de optimización automática. Además en clases de
prácticas hemos visto como interpretar la información del fichero pero no como mejorar manualmente un bucle para su
vectorización, entonces no se si es porque realmente no se puede o porque no se ha planteado durante las sesiones.
\par En la parte de vectorización ha sido la más difícil porque probando en tres máquinas diferentes me salían distintos resultados. Las
diapositivas de ayuda en inglés solo me han servido para liarme más con algunos conceptos y he echado en falta un ejemplo en
clase que fusionara todos los mecanismos diferentes que se pueden integrar para paralelizar. Ha servido de ayuda que dejaran una
semana más porque para el primer ejercicio necesitaba parte de la teoría que se daba al final del tema correspondiente. Me gustaría
que dejaran la solución correcta en el aula virtual porque no sé si algunos razonamientos o implementaciones son correctos o si
puede haber una mejor versión. 