%%%%%%%%%%%%%%%%%%%%%% Procesadores CMP: usando la vectorización y la paralelización - EJERCICIO 2 %%%%%%%%%%%%%%%%%%%%%%
\newpage
\section{Ejercicio 2}
\subsection{Enunciado}
\begin{ejer}
    \textbf{2.-} En el directorio \texttt{paralelización} hay un pequeño programa que simula la propagación del 
    calor en una superficie rectangular. Se supone que la temperatura de cada uno de los bordes de la
    superficie se mantiene constante y el programa calcula la temperatura final de cada punto de
    la superficie representada como una matriz. Para ello utiliza un algoritmo iterativo tipo \texttt{stencil},
    el cual actualiza cada uno de los elementos de la matriz en función del valor previo del mismo
    elemento y sus vecinos.
    \par El código principal del algoritmo se encuentra en el fichero \texttt{heat.cpp}, que será el único que se
    necesitará modificar. Además, el fichero \texttt{matrix.h} proporciona un tipo de datos para almacenar las
    matrices utilizadas por el algoritmo, mientras que el fichero \texttt{main.cpp} incluye el código necesario
    para ejecutar el programa y medir el tiempo empleado en realizar la simulación. El programa permite
     configurar múltiples parámetros de la ejecución mediante opciones de la línea de comandos 
     (ver \texttt{main.cpp}).
    \par Además del programa, se incluye un \texttt{Makefile} para compilarlo usando tanto el compilador \texttt{GCC}
    como el \texttt{ICC}. Se incluyen también scripts para automatizar la comprobación de que los resultados
    son correctos y para automatizar la medida del tiempo de ejecución del programa usando diferente
    número de hilos (generando ficheros \texttt{tsv} que se pueden utilizar para generar gráficas fácilmente
    con multitud de programas, incluyendo cualquier hoja de cálculo). Se aconseja el uso de un
    ordenador que tenga 4 cores o más. Finalmente, se incluyen tres casos de prueba y su salida
    correspondiente, que se utilizarán para comprobar la corrección del programa modificado y para
    medir su rendimiento. El \texttt{Makefile} incluye objetivos para ejecutar todos los tests
    (\texttt{make tests} o \texttt{make tests-icc}) y para realizar la toma de tiempos (\texttt{make times} o
     \texttt{make times-icc}).
    \par Para completar este ejercicio, realiza lo siguiente:
    \begin{itemize}
        \item El procedimiento \texttt{solve} de \texttt{heat.cpp} tiene 3 bucles. Identifícalos y explica para cada uno
        de ellos si es candidato para la paralelización usando OpenMP. Indica en cada caso qué
        \texttt{pragma} (o \texttt{pragmas}) sería necesario añadir para realizar la paralelización correctamente.
        Presta especial cuidado a la clasificación de las variables (ya sean de reducción, compartidas
        o privadas).
        \item Prueba a paralelizar individualmente cada uno de los bucles paralelizables y prueba también a
        paralelizar todos los bucles paralelizables a la vez. Para cada versión resultante del programa,
        mide el tiempo de ejecución con diferente número de hilos de cada uno de los tres casos de
        prueba incluidos y genera las correspondientes gráficas mostrando el tiempo de ejecución y
        la escalabilidad obtenida (similar a las figuras \ref{fig:p2enun2-1} y \ref{fig:p2enun2-2}). Comenta los resultados.
    \end{itemize}
\end{ejer}
 %%% IMÁGENES %%%
\begin{figure}
    \centering
    \begin{subfigure}{0.4\textwidth}
        \includegraphics[width=\textwidth]{p2enun2-1}
        \caption{Tiempo de ejecución vs nº de hilos}
        \label{fig:p2enun2-1}
    \end{subfigure}
    \begin{subfigure}{0.4\textwidth}
        \includegraphics[width=\textwidth]{p2enun2-2}
        \caption{Escalabilidad de la aplicación}
        \label{fig:p2enun2-2}
    \end{subfigure}
    \caption{Gráficas enunciado}
\end{figure}

\subsection{Tiempos y gráficas}


\subsubsection{\textbf{Código inicial sin paralelización:}}

\par En estas tres tablas y gráficas (\ref{fig:sin-cambios=01-small}, \ref{sin-cambios=02-medium} y \ref{sin-cambios=03-large}) se muestran los tiempos de ejecución dependiendo del número de hilos,
 tipo de compilador y el tamaño del problema.

\definecolor{azul}{rgb}{0.36, 0.54, 0.66}

%%% TABLA DE TIEMPOS E IMÁGENES %%%
\begin{figure}[H]
    \centering
    \begin{subfigure}{0.4\textwidth}
        \begin{adjustbox}{width=\textwidth} 
        \begin{tabular}{|c|c|c|c|c|}
            \hline
            \rowcolor{azul} \multicolumn{2}{|c|}{}&\multicolumn{3}{c|}{\textbf{Compiler}} \\ \hline
            \rowcolor{azul} \multicolumn{2}{|c|}{}&\texttt{clang}&\texttt{gcc}&\texttt{icc}\\ \hline
            \rowcolor{azul} \textbf{Testing size} & \textbf{Threads}&\multicolumn{3}{c|}{\textbf{Average time (s)}} \\ \hline
            \multirow{8}{1cm}{\textbf{01-small}} & 1 & \(0.83\pm{0.01}\) & \(0.39\pm{0.02}\) & \(1.27\pm{0.06}\) \\ \cline{2-5}
            & 2 & \(0.86\pm{0.03}\) & \(0.34\pm{0.02}\) & \(1.49\pm{0.42}\) \\ \cline{2-5}
            & 3 & \(0.86\pm{0.03}\) & \(0.42\pm{0.05}\) & \(1.02\pm{0.00}\) \\ \cline{2-5}
            & 4 & \(0.84\pm{0.01}\) & \(0.46\pm{0.08}\) & \(1.02\pm{0.00}\) \\ \cline{2-5}
            & 5 & \(0.83\pm{0.01}\) & \(0.45\pm{0.01}\) & \(1.01\pm{0.00}\) \\ \cline{2-5}
            & 6 & \(0.83\pm{0.01}\) & \(0.39\pm{0.02}\) & \(1.01\pm{0.00}\) \\ \cline{2-5}
            & 7 & \(0.83\pm{0.01}\) & \(0.40\pm{0.01}\) & \(1.02\pm{0.00}\) \\ \cline{2-5}
            & 8 & \(0.83\pm{0.01}\) & \(0.40\pm{0.01}\) & \(1.05\pm{0.02}\) \\ \hline
        \end{tabular}
        \end{adjustbox}
    \end{subfigure}
    \hfill
    \begin{subfigure}{0.5\textwidth}
        \includegraphics[width=\textwidth]{sin-cambios=01-small}
    \end{subfigure}
    \caption{\underline{Código inicial, tamaño pequeño}: Tiempos de ejecución vs nº de hilos}
    \label{fig:sin-cambios=01-small}
\end{figure}

%%% TABLA DE TIEMPOS E IMÁGENES %%%
\begin{figure}[H]
    \centering
    \begin{subfigure}{0.4\textwidth}
        \begin{adjustbox}{width=\textwidth} 
        \begin{tabular}{|c|c|c|c|c|}
            \hline
            \rowcolor{azul} \multicolumn{2}{|c|}{}&\multicolumn{3}{c|}{\textbf{Compiler}} \\ \hline
            \rowcolor{azul} \multicolumn{2}{|c|}{}&\texttt{clang}&\texttt{gcc}&\texttt{icc}\\ \hline
            \rowcolor{azul} \textbf{Testing size} & \textbf{Threads}&\multicolumn{3}{c|}{\textbf{Average time (s)}} \\ \hline
            \multirow{8}{2.5cm}{\textbf{02-medium}} & 1 & \(2.38\pm{0.02}\) & \(1.15\pm{0.25}\) & \(2.91\pm{0.07}\) \\ \cline{2-5}
            & 2 & \(2.47\pm{0.10}\) & \(1.18\pm{0.24}\) & \(3.37\pm{0.39}\) \\ \cline{2-5}
            & 3 & \(2.38\pm{0.02}\) & \(1.33\pm{0.44}\) & \(2.93\pm{0.08}\) \\ \cline{2-5}
            & 4 & \(2.41\pm{0.05}\) & \(1.20\pm{0.31}\) & \(2.97\pm{0.12}\) \\ \cline{2-5}
            & 5 & \(2.42\pm{0.05}\) & \(1.20\pm{0.30}\) & \(2.96\pm{0.12}\) \\ \cline{2-5}
            & 6 & \(2.43\pm{0.06}\) & \(1.19\pm{0.29}\) & \(2.96\pm{0.12}\) \\ \cline{2-5}
            & 7 & \(2.38\pm{0.02}\) & \(1.87\pm{0.34}\) & \(2.93\pm{0.07}\) \\ \cline{2-5}
            & 8 & \(2.39\pm{0.03}\) & \(1.40\pm{0.25}\) & \(2.92\pm{0.07}\) \\ \hline
        \end{tabular}
        \end{adjustbox}
    \end{subfigure}
    \hfill
    \begin{subfigure}{0.5\textwidth}
        \includegraphics[width=\textwidth]{sin-cambios=02-medium}
    \end{subfigure}
    \caption{\underline{Código inicial, tamaño mediano}: Tiempos de ejecución vs nº de hilos}
    \label{sin-cambios=02-medium}
\end{figure}

%%% TABLA DE TIEMPOS E IMÁGENES %%%
\begin{figure}[H]
    \centering
    \begin{subfigure}{0.4\textwidth}
        \begin{adjustbox}{width=\textwidth} 
        \begin{tabular}{|c|c|c|c|c|}
            \hline
            \rowcolor{azul} \multicolumn{2}{|c|}{}&\multicolumn{3}{c|}{\textbf{Compiler}} \\ \hline
            \rowcolor{azul} \multicolumn{2}{|c|}{}&\texttt{clang}&\texttt{gcc}&\texttt{icc}\\ \hline
            \rowcolor{azul} \textbf{Testing size} & \textbf{Threads}&\multicolumn{3}{c|}{\textbf{Average time (s)}} \\ \hline
            \multirow{8}{1cm}{\textbf{03-large}} & 1 & \(4.12\pm{0.08}\) & \(1.61\pm{0.12}\) & \(4.98\pm{0.15}\) \\ \cline{2-5}
            & 2 & \(4.10\pm{0.06}\) & \(1.61\pm{0.11}\) & \(4.94\pm{0.12}\) \\ \cline{2-5}
            & 3 & \(4.13\pm{0.09}\) & \(1.61\pm{0.12}\) & \(5.04\pm{0.20}\) \\ \cline{2-5}
            & 4 & \(4.10\pm{0.06}\) & \(1.60\pm{0.11}\) & \(4.95\pm{0.11}\) \\ \cline{2-5}
            & 5 & \(4.10\pm{0.06}\) & \(1.63\pm{0.12}\) & \(4.98\pm{0.15}\) \\ \cline{2-5}
            & 6 & \(4.16\pm{0.13}\) & \(1.60\pm{0.12}\) & \(4.94\pm{0.11}\) \\ \cline{2-5}
            & 7 & \(4.17\pm{0.11}\) & \(1.59\pm{0.12}\) & \(4.98\pm{0.15}\) \\ \cline{2-5}
            & 8 & \(4.15\pm{0.11}\) & \(1.60\pm{0.11}\) & \(4.95\pm{0.12}\) \\ \hline
        \end{tabular}
        \end{adjustbox}
    \end{subfigure}
    \hfill
    \begin{subfigure}{0.5\textwidth}
        \includegraphics[width=\textwidth]{sin-cambios=03-large}
    \end{subfigure}
    \caption{\underline{Código inicial, tamaño largo}: Tiempos de ejecución vs nº de hilos}
    \label{sin-cambios=03-large}
\end{figure}

\subsubsection{\textbf{1ºBucle:}}
\begin{listing}
    @@ -20,3 +20,4 @@
\end{listing}


\subsubsection{\textbf{2ºBucle:}}
\begin{listing}[firstnumber=21]
    @@ -22,4 +22,4 @@
    for (int i = 0; i < arrays_size - 1; ++i) {
      A[i] = A[i] - B[i];
      B[i + 1] = C[i] * 2;
    }
\end{listing}
\par Analizando la ejecución desarrollada del bucle se pueden ver las dependencias:
\begin{listing}[numbers=none]
    // Dentro del bucle
    A[0] = A[0] - B[0];
    B[1] = C[0] * 2;

    A[1] = A[1] - B[1];
    B[2] = C[1] * 2;

            ...

    A[arrays_size-1] = A[arrays_size-1] - B[arrays_size-1];
    B[arrays_size] = C[arrays_size-1] * 2; 
\end{listing}
\par En este bucle se producen dependencias \textbf{RAW} porque en la línea 23 se quiere calcular \texttt{A[i]}, pero depende del valor \texttt{B[i]} obtenido
en la anterior iteración. Este caso es mucho más sencillo, la dependencia no afectaría a la vectorización porque el compilador lo
resolvería reordenando las instrucciones, internamente en código ensamblador, el bucle se transformaría en el siguiente código
para eliminar la dependencia.
\begin{listing}[numbers=none]
    // Fuera del bucle
    A[0] = A[0] - B[0];
    // Dentro del bucle
    B[1] = C[0] * 2;
    A[1] = A[1] - B[1];

    B[2] = C[1] * 2;
    A[2] = A[2] - B[2];

            ...
    B[arrays_size-1] = C[arrays_size-2] * 2;
    A[arrays_size-1] = A[arrays_size-1] - B[arrays_size-1];
    //Fuera del bucle
    B[arrays_size] = C[arrays_size-1] * 2; 
\end{listing}
\begin{listing}[firstnumber=21]
    @@ -22,4 +22,6 @@
    + A[0] = A[0] - B[0];
    + for (int i = 0; i < arrays_size - 2; ++i) {
    +    B[i + 1] = C[i] * 2;
    +    A[i] = A[i] - B[i];
    }
    + B[arrays_size] = C[arrays_size-1] * 2;
\end{listing}

\subsubsection{\textbf{3ºBucle:}}

\par El tercer bucle es el bucle interior que está anidado \texttt{for (size\_t j = 1; j < state.width - 1; ++j)},
 se añadiría el mismo \texttt{pragma} que en el caso anterior pero el rendimiento es mucho más lento.

%%% TABLA DE TIEMPOS E IMÁGENES %%%
\begin{figure}[H]
    \centering
    \begin{subfigure}{0.4\textwidth}
        \begin{adjustbox}{width=\textwidth} 
        \begin{tabular}{|c|c|c|c|c|}
            \hline
            \rowcolor{azul} \multicolumn{2}{|c|}{}&\multicolumn{3}{c|}{\textbf{Compiler}} \\ \hline
            \rowcolor{azul} \multicolumn{2}{|c|}{}&\texttt{clang}&\texttt{gcc}&\texttt{icc}\\ \hline
            \rowcolor{azul} \textbf{Testing size} & \textbf{Threads}&\multicolumn{3}{c|}{\textbf{Average time (s)}} \\ \hline
            \multirow{8}{1cm}{\textbf{01-small}} & 1 & \(1.98\pm{0.00}\) & \(1.30\pm{0.14}\) & \(1.52\pm{0.01}\) \\ \cline{2-5}
            & 2 & \(2.78\pm{0.03}\) & \(1.82\pm{0.01}\) & \(3.07\pm{0.00}\) \\ \cline{2-5}
            & 3 & \(2.72\pm{0.01}\) & \(2.22\pm{0.01}\) & \(3.28\pm{0.02}\) \\ \cline{2-5}
            & 4 & \(3.04\pm{0.01}\) & \(2.79\pm{0.00}\) & \(3.70\pm{0.01}\) \\ \cline{2-5}
            & 5 & \(4.53\pm{0.02}\) & \(3.04\pm{0.00}\) & \(4.44\pm{0.03}\) \\ \cline{2-5}
            & 6 & \(4.58\pm{0.01}\) & \(3.51\pm{0.01}\) & \(4.66\pm{0.05}\) \\ \cline{2-5}
            & 7 & \(5.09\pm{0.00}\) & \(3.80\pm{0.01}\) & \(5.22\pm{0.00}\) \\ \cline{2-5}
            & 8 & \(5.41\pm{0.04}\) & \(4.35\pm{0.02}\) & \(5.74\pm{0.00}\) \\ \hline
        \end{tabular}
        \end{adjustbox}
    \end{subfigure}
    \hfill
    \begin{subfigure}{0.5\textwidth}
        \includegraphics[width=\textwidth]{bucle3=01-small}
    \end{subfigure}
    \caption{\underline{Tamaño pequeño}: Tiempos de ejecución vs nº de hilos}
    \label{fig:bucle3=01-small}
\end{figure}

%%% TABLA DE TIEMPOS E IMÁGENES %%%
\begin{figure}[H]
    \centering
    \begin{subfigure}{0.4\textwidth}
        \begin{adjustbox}{width=\textwidth} 
        \begin{tabular}{|c|c|c|c|c|}
            \hline
            \rowcolor{azul} \multicolumn{2}{|c|}{}&\multicolumn{3}{c|}{\textbf{Compiler}} \\ \hline
            \rowcolor{azul} \multicolumn{2}{|c|}{}&\texttt{clang}&\texttt{gcc}&\texttt{icc}\\ \hline
            \rowcolor{azul} \textbf{Testing size} & \textbf{Threads}&\multicolumn{3}{c|}{\textbf{Average time (s)}} \\ \hline
            \multirow{8}{2.5cm}{\textbf{02-medium}} & 1 & \(5.29\pm{0.01}\) & \(2.05\pm{0.06}\) & \(3.87\pm{0.04}\) \\ \cline{2-5}
            & 2 & \(6.35\pm{0.02}\) & \(3.59\pm{0.03}\) & \(6.36\pm{0.05}\) \\ \cline{2-5}
            & 3 & \(5.78\pm{0.00}\) & \(4.24\pm{0.02}\) & \(6.71\pm{0.03}\) \\ \cline{2-5}
            & 4 & \(6.21\pm{0.01}\) & \(5.30\pm{0.02}\) & \(7.54\pm{0.04}\) \\ \cline{2-5}
            & 5 & \(8.94\pm{0.05}\) & \(6.17\pm{0.00}\) & \(9.16\pm{0.02}\) \\ \cline{2-5}
            & 6 & \(9.56\pm{0.19}\) & \(6.69\pm{0.01}\) & \(9.43\pm{0.04}\) \\ \cline{2-5}
            & 7 & \(10.31\pm{0.06}\) & \(7.64\pm{0.00}\) & \(10.33\pm{0.00}\) \\ \cline{2-5}
            & 8 & \(11.27\pm{0.00}\) & \(8.66\pm{0.03}\) & \(11.21\pm{0.02}\) \\ \hline
        \end{tabular}
        \end{adjustbox}
    \end{subfigure}
    \hfill
    \begin{subfigure}{0.5\textwidth}
        \includegraphics[width=\textwidth]{bucle3=02-medium}
    \end{subfigure}
    \caption{\underline{Tamaño mediano}: Tiempos de ejecución vs nº de hilos}
    \label{bucle3=02-medium}
\end{figure}

%%% TABLA DE TIEMPOS E IMÁGENES %%%
\begin{figure}[H]
    \centering
    \begin{subfigure}{0.4\textwidth}
        \begin{adjustbox}{width=\textwidth} 
        \begin{tabular}{|c|c|c|c|c|}
            \hline
            \rowcolor{azul} \multicolumn{2}{|c|}{}&\multicolumn{3}{c|}{\textbf{Compiler}} \\ \hline
            \rowcolor{azul} \multicolumn{2}{|c|}{}&\texttt{clang}&\texttt{gcc}&\texttt{icc}\\ \hline
            \rowcolor{azul} \textbf{Testing size} & \textbf{Threads}&\multicolumn{3}{c|}{\textbf{Average time (s)}} \\ \hline
            \multirow{8}{1cm}{\textbf{03-large}} & 1 & \(8.62\pm{0.04}\) & \(3.03\pm{0.11}\) & \(6.27\pm{0.16}\) \\ \cline{2-5}
            & 2 & \(8.98\pm{0.06}\) & \(3.03\pm{0.09}\) & \(8.94\pm{0.08}\) \\ \cline{2-5}
            & 3 & \(8.15\pm{0.07}\) & \(4.77\pm{0.05}\) & \(9.04\pm{0.11}\) \\ \cline{2-5}
            & 4 & \(8.25\pm{0.03}\) & \(6.09\pm{0.04}\) & \(9.92\pm{0.00}\) \\ \cline{2-5}
            & 5 & \(11.96\pm{0.09}\) & \(7.25\pm{0.02}\) & \(12.09\pm{0.08}\) \\ \cline{2-5}
            & 6 & \(12.23\pm{0.09}\) & \(8.06\pm{0.02}\) & \(12.84\pm{0.04}\) \\ \cline{2-5}
            & 7 & \(13.77\pm{0.04}\) & \(9.78\pm{0.03}\) & \(13.59\pm{0.05}\) \\ \cline{2-5}
            & 8 & \(14.93\pm{0.04}\) & \(10.70\pm{0.03}\) & \(14.98\pm{0.08}\) \\ \hline
        \end{tabular}
        \end{adjustbox}
    \end{subfigure}
    \hfill
    \begin{subfigure}{0.5\textwidth}
        \includegraphics[width=\textwidth]{bucle3=03-large}
    \end{subfigure}
    \caption{\underline{Tamaño largo}: Tiempos de ejecución vs nº de hilos}
    \label{bucle3=03-large}
\end{figure}

\par Estos dos bucles anidados también podría ponerse el pragma \texttt{\#pragma omp parallel for collapse(2) reduction (max:difference)}
pero comprobando los tiempos se puede ver que empeora el tiempo de ejecución.


\subsubsection{\textbf{Paralelización de los tres bucles:}}
\subsubsection{\textbf{Bucle 1+2:}}
\par Esta combinación se asemeja a los resultados de la opción de solo intentar paralelizar el primer 
bucle, por tanto en algunos casos mejora pero insignitivamente. Además, cuantos más hilos hay, empeora 
el tiempo de ejecución progresivamente.
%%% TABLA DE TIEMPOS E IMÁGENES %%%
\begin{figure}[H]
    \centering
    \begin{subfigure}{0.4\textwidth}
        \begin{adjustbox}{width=\textwidth} 
        \begin{tabular}{|c|c|c|c|c|}
            \hline
            \rowcolor{azul} \multicolumn{2}{|c|}{}&\multicolumn{3}{c|}{\textbf{Compiler}} \\ \hline
            \rowcolor{azul} \multicolumn{2}{|c|}{}&\texttt{clang}&\texttt{gcc}&\texttt{icc}\\ \hline
            \rowcolor{azul} \textbf{Testing size} & \textbf{Threads}&\multicolumn{3}{c|}{\textbf{Average time (s)}} \\ \hline
            \multirow{8}{1cm}{\textbf{01-small}} & 1 & \(1.53\pm{0.00}\) & \(0.38\pm{0.01}\) & \(1.01\pm{0.0}\) \\ \cline{2-5}
            & 2 & \(1.56\pm{0.00}\) & \(0.35\pm{0.01}\) & \(1.03\pm{0.01}\) \\ \cline{2-5}
            & 3 & \(1.58\pm{0.00}\) & \(0.34\pm{0.02}\) & \(1.06\pm{0.05}\) \\ \cline{2-5}
            & 4 & \(1.60\pm{0.00}\) & \(0.34\pm{0.02}\) & \(1.08\pm{0.01}\) \\ \cline{2-5}
            & 5 & \(1.73\pm{0.12}\) & \(0.47\pm{0.01}\) & \(1.42\pm{0.35}\) \\ \cline{2-5}
            & 6 & \(1.70\pm{0.10}\) & \(0.48\pm{0.01}\) & \(1.42\pm{0.35}\) \\ \cline{2-5}
            & 7 & \(1.73\pm{0.08}\) & \(0.48\pm{0.01}\) & \(1.46\pm{0.41}\) \\ \cline{2-5}
            & 8 & \(1.83\pm{0.02}\) & \(0.47\pm{0.01}\) & \(1.79\pm{0.02}\) \\ \hline
        \end{tabular}
        \end{adjustbox}
    \end{subfigure}
    \hfill
    \begin{subfigure}{0.5\textwidth}
        \includegraphics[width=\textwidth]{bucle1-2=01-small}
    \end{subfigure}
    \caption{\underline{Tamaño pequeño}: Tiempos de ejecución vs nº de hilos}
    \label{fig:bucle1-2=01-small}
\end{figure}

%%% TABLA DE TIEMPOS E IMÁGENES %%%
\begin{figure}[H]
    \centering
    \begin{subfigure}{0.4\textwidth}
        \begin{adjustbox}{width=\textwidth} 
        \begin{tabular}{|c|c|c|c|c|}
            \hline
            \rowcolor{azul} \multicolumn{2}{|c|}{}&\multicolumn{3}{c|}{\textbf{Compiler}} \\ \hline
            \rowcolor{azul} \multicolumn{2}{|c|}{}&\texttt{clang}&\texttt{gcc}&\texttt{icc}\\ \hline
            \rowcolor{azul} \textbf{Testing size} & \textbf{Threads}&\multicolumn{3}{c|}{\textbf{Average time (s)}} \\ \hline
            \multirow{8}{2.5cm}{\textbf{02-medium}} & 1 & \(4.50\pm{0.09}\) & \(0.88\pm{0.04}\) & \(2.94\pm{0.08}\) \\ \cline{2-5}
            & 2 & \(4.54\pm{0.03}\) & \(0.91\pm{0.04}\) & \(2.96\pm{0.04}\) \\ \cline{2-5}
            & 3 & \(4.53\pm{0.01}\) & \(0.91\pm{0.04}\) & \(2.97\pm{0.05}\) \\ \cline{2-5}
            & 4 & \(4.64\pm{0.02}\) & \(0.93\pm{0.04}\) & \(3.02\pm{0.04}\) \\ \cline{2-5}
            & 5 & \(4.68\pm{0.05}\) & \(0.94\pm{0.04}\) & \(3.03\pm{0.04}\) \\ \cline{2-5}
            & 6 & \(5.20\pm{0.02}\) & \(1.31\pm{0.03}\) & \(3.03\pm{0.04}\) \\ \cline{2-5}
            & 7 & \(5.25\pm{0.04}\) & \(1.30\pm{0.02}\) & \(5.15\pm{0.05}\) \\ \cline{2-5}
            & 8 & \(5.21\pm{0.00}\) & \(1.31\pm{0.03}\) & \(5.13\pm{0.06}\) \\ \hline
        \end{tabular}
        \end{adjustbox}
    \end{subfigure}
    \hfill
    \begin{subfigure}{0.5\textwidth}
        \includegraphics[width=\textwidth]{bucle1-2=02-medium}
    \end{subfigure}
    \caption{\underline{Tamaño mediano}: Tiempos de ejecución vs nº de hilos}
    \label{bucle1-2=02-medium}
\end{figure}

%%% TABLA DE TIEMPOS E IMÁGENES %%%
\begin{figure}[H]
    \centering
    \begin{subfigure}{0.4\textwidth}
        \begin{adjustbox}{width=\textwidth} 
        \begin{tabular}{|c|c|c|c|c|}
            \hline
            \rowcolor{azul} \multicolumn{2}{|c|}{}&\multicolumn{3}{c|}{\textbf{Compiler}} \\ \hline
            \rowcolor{azul} \multicolumn{2}{|c|}{}&\texttt{clang}&\texttt{gcc}&\texttt{icc}\\ \hline
            \rowcolor{azul} \textbf{Testing size} & \textbf{Threads}&\multicolumn{3}{c|}{\textbf{Average time (s)}} \\ \hline
            \multirow{8}{1cm}{\textbf{03-large}} & 1 & \(7.56\pm{0.02}\) & \(1.53\pm{0.08}\) & \(5.00\pm{0.11}\) \\ \cline{2-5}
            & 2 & \(7.78\pm{0.01}\) & \(1.57\pm{0.08}\) & \(5.07\pm{0.10}\) \\ \cline{2-5}
            & 3 & \(7.80\pm{0.04}\) & \(1.57\pm{0.08}\) & \(5.18\pm{0.21}\) \\ \cline{2-5}
            & 4 & \(7.92\pm{0.06}\) & \(1.61\pm{0.07}\) & \(5.22\pm{0.12}\) \\ \cline{2-5}
            & 5 & \(7.85\pm{0.02}\) & \(2.20\pm{0.04}\) & \(5.31\pm{0.19}\) \\ \cline{2-5}
            & 6 & \(8.89\pm{0.06}\) & \(1.62\pm{0.07}\) & \(5.16\pm{0.09}\) \\ \cline{2-5}
            & 7 & \(8.90\pm{0.02}\) & \(2.21\pm{0.04}\) & \(8.71\pm{0.04}\) \\ \cline{2-5}
            & 8 & \(8.83\pm{0.03}\) & \(2.21\pm{0.04}\) & \(8.75\pm{0.14}\) \\ \hline
        \end{tabular}
        \end{adjustbox}
    \end{subfigure}
    \hfill
    \begin{subfigure}{0.5\textwidth}
        \includegraphics[width=\textwidth]{bucle1-2=03-large}
    \end{subfigure}
    \caption{\underline{Tamaño largo}: Tiempos de ejecución vs nº de hilos}
    \label{bucle1-2=03-large}
\end{figure}

%%%%%%%%%%%%%%%%%%%%%%%%%%%%%%%%%%%%%%%%%%%%%%%%%%%%%%%%%%%%%%%%%%%%%%%%%%%%%%%%%
\subsubsection{\textbf{Bucle 2+3:}}

\par Estos dos bucles anidados también podría ponerse el pragma \texttt{\#pragma omp parallel for collapse(2) reduction (+:difference)}
antes del primer bucle y comprobando los tiempos se puede ver que empeora el tiempo de ejecución considerablemente. Esta combinación se asemeja a los resultados de 
la opción más óptima en cuanto a la trayectoria de la gráfica, pero si comparamos con los resultados de la ejecución del programa sin 
modificaciones podemos ver que esta combinación no mejora nada y además empeora cuando hay pocos hilos. 

%%% TABLA DE TIEMPOS E IMÁGENES %%%
\begin{figure}[H]
    \centering
    \begin{subfigure}{0.4\textwidth}
        \begin{adjustbox}{width=\textwidth} 
        \begin{tabular}{|c|c|c|c|c|}
            \hline
            \rowcolor{azul} \multicolumn{2}{|c|}{}&\multicolumn{3}{c|}{\textbf{Compiler}} \\ \hline
            \rowcolor{azul} \multicolumn{2}{|c|}{}&\texttt{clang}&\texttt{gcc}&\texttt{icc}\\ \hline
            \rowcolor{azul} \textbf{Testing size} & \textbf{Threads}&\multicolumn{3}{c|}{\textbf{Average time (s)}} \\ \hline
            \multirow{8}{1cm}{\textbf{01-small}} & 1 & \(3.36\pm{0.02}\) & \(1.80\pm{0.38}\) & \(5.68\pm{0.01}\) \\ \cline{2-5}
            & 2 & \(1.74\pm{0.01}\) & \(0.72\pm{0.01}\) & \(2.93\pm{0.01}\) \\ \cline{2-5}
            & 3 & \(1.19\pm{0.02}\) & \(0.50\pm{0.01}\) & \(1.98\pm{0.01}\) \\ \cline{2-5}
            & 4 & \(0.91\pm{0.01}\) & \(0.39\pm{0.01}\) & \(1.52\pm{0.01}\) \\ \cline{2-5}
            & 5 & \(1.35\pm{0.00}\) & \(0.59\pm{0.00}\) & \(2.21\pm{0.00}\) \\ \cline{2-5}
            & 6 & \(1.13\pm{0.00}\) & \(0.49\pm{0.00}\) & \(1.85\pm{0.01}\) \\ \cline{2-5}
            & 7 & \(0.98\pm{0.01}\) & \(0.43\pm{0.00}\) & \(1.60\pm{0.01}\) \\ \cline{2-5}
            & 8 & \(0.87\pm{0.00}\) & \(0.40\pm{0.01}\) & \(1.44\pm{0.03}\) \\ \hline
        \end{tabular}
        \end{adjustbox}
    \end{subfigure}
    \hfill
    \begin{subfigure}{0.5\textwidth}
        \includegraphics[width=\textwidth]{bucle2-3=01-small}
    \end{subfigure}
    \caption{\underline{Tamaño pequeño}: Tiempos de ejecución vs nº de hilos}
    \label{fig:bucle2-3=01-small}
\end{figure}

%%% TABLA DE TIEMPOS E IMÁGENES %%%
\begin{figure}[H]
    \centering
    \begin{subfigure}{0.4\textwidth}
        \begin{adjustbox}{width=\textwidth} 
        \begin{tabular}{|c|c|c|c|c|}
            \hline
            \rowcolor{azul} \multicolumn{2}{|c|}{}&\multicolumn{3}{c|}{\textbf{Compiler}} \\ \hline
            \rowcolor{azul} \multicolumn{2}{|c|}{}&\texttt{clang}&\texttt{gcc}&\texttt{icc}\\ \hline
            \rowcolor{azul} \textbf{Testing size} & \textbf{Threads}&\multicolumn{3}{c|}{\textbf{Average time (s)}} \\ \hline
            \multirow{8}{2.5cm}{\textbf{02-medium}} & 1 & \(9.68\pm{0.02}\) & \(4.00\pm{0.03}\) & \(16.34\pm{0.03}\) \\ \cline{2-5}
            & 2 & \(4.99\pm{0.03}\) & \(2.06\pm{0.03}\) & \(8.40\pm{0.06}\) \\ \cline{2-5}
            & 3 & \(3.36\pm{0.02}\) & \(1.45\pm{0.08}\) & \(5.65\pm{0.01}\) \\ \cline{2-5}
            & 4 & \(2.59\pm{0.03}\) & \(1.12\pm{0.06}\) & \(4.33\pm{0.02}\) \\ \cline{2-5}
            & 5 & \(3.84\pm{0.00}\) & \(1.66\pm{0.01}\) & \(6.32\pm{0.01}\) \\ \cline{2-5}
            & 6 & \(3.21\pm{0.00}\) & \(1.39\pm{0.00}\) & \(5.28\pm{0.00}\) \\ \cline{2-5}
            & 7 & \(2.76\pm{0.00}\) & \(1.22\pm{0.02}\) & \(4.54\pm{0.01}\) \\ \cline{2-5}
            & 8 & \(2.47\pm{0.01}\) & \(1.09\pm{0.02}\) & \(4.39\pm{0.08}\) \\ \hline
        \end{tabular}
        \end{adjustbox}
    \end{subfigure}
    \hfill
    \begin{subfigure}{0.5\textwidth}
        \includegraphics[width=\textwidth]{bucle2-3=02-medium}
    \end{subfigure}
    \caption{\underline{Tamaño mediano}: Tiempos de ejecución vs nº de hilos}
    \label{bucle2-3=02-medium}
\end{figure}

%%% TABLA DE TIEMPOS E IMÁGENES %%%
\begin{figure}[H]
    \centering
    \begin{subfigure}{0.4\textwidth}
        \begin{adjustbox}{width=\textwidth} 
        \begin{tabular}{|c|c|c|c|c|}
            \hline
            \rowcolor{azul} \multicolumn{2}{|c|}{}&\multicolumn{3}{c|}{\textbf{Compiler}} \\ \hline
            \rowcolor{azul} \multicolumn{2}{|c|}{}&\texttt{clang}&\texttt{gcc}&\texttt{icc}\\ \hline
            \rowcolor{azul} \textbf{Testing size} & \textbf{Threads}&\multicolumn{3}{c|}{\textbf{Average time (s)}} \\ \hline
            \multirow{8}{1cm}{\textbf{03-large}} & 1 & \(16.48\pm{0.01}\) & \(6.96\pm{0.12}\) & \(27.84\pm{0.06}\) \\ \cline{2-5}
            & 2 & \(8.54\pm{0.04}\) & \(3.55\pm{0.10}\) & \(14.33\pm{0.07}\) \\ \cline{2-5}
            & 3 & \(5.75\pm{0.00}\) & \(2.39\pm{0.03}\) & \(9.61\pm{0.06}\) \\ \cline{2-5}
            & 4 & \(4.41\pm{0.00}\) & \(1.85\pm{0.06}\) & \(7.35\pm{0.00}\) \\ \cline{2-5}
            & 5 & \(6.60\pm{0.05}\) & \(2.82\pm{0.01}\) & \(10.79\pm{0.02}\) \\ \cline{2-5}
            & 6 & \(5.48\pm{0.01}\) & \(2.36\pm{0.01}\) & \(8.99\pm{0.01}\) \\ \cline{2-5}
            & 7 & \(4.69\pm{0.02}\) & \(2.03\pm{0.01}\) & \(7.72\pm{0.00}\) \\ \cline{2-5}
            & 8 & \(4.16\pm{0.00}\) & \(1.82\pm{0.03}\) & \(6.90\pm{0.04}\) \\ \hline
        \end{tabular}
        \end{adjustbox}
    \end{subfigure}
    \hfill
    \begin{subfigure}{0.5\textwidth}
        \includegraphics[width=\textwidth]{bucle2-3=03-large}
    \end{subfigure}
    \caption{\underline{Tamaño largo}: Tiempos de ejecución vs nº de hilos}
    \label{bucle2-3=03-large}
\end{figure}



