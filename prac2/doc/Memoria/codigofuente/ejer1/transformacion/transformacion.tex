\subsection{Transformación manual para vectorizar}

\subsubsection{\textbf{1ºBucle:}}
\begin{listing}
    @@ -20,3 +20,4 @@
\end{listing}


\newpage

\subsubsection{\textbf{3ºBucle:}}

\par El tercer bucle es el bucle interior que está anidado \texttt{for (size\_t j = 1; j < state.width - 1; ++j)},
 se añadiría el mismo \texttt{pragma} que en el caso anterior pero el rendimiento es mucho más lento.

%%% TABLA DE TIEMPOS E IMÁGENES %%%
\begin{figure}[H]
    \centering
    \begin{subfigure}{0.4\textwidth}
        \begin{adjustbox}{width=\textwidth} 
        \begin{tabular}{|c|c|c|c|c|}
            \hline
            \rowcolor{azul} \multicolumn{2}{|c|}{}&\multicolumn{3}{c|}{\textbf{Compiler}} \\ \hline
            \rowcolor{azul} \multicolumn{2}{|c|}{}&\texttt{clang}&\texttt{gcc}&\texttt{icc}\\ \hline
            \rowcolor{azul} \textbf{Testing size} & \textbf{Threads}&\multicolumn{3}{c|}{\textbf{Average time (s)}} \\ \hline
            \multirow{8}{1cm}{\textbf{01-small}} & 1 & \(1.98\pm{0.00}\) & \(1.30\pm{0.14}\) & \(1.52\pm{0.01}\) \\ \cline{2-5}
            & 2 & \(2.78\pm{0.03}\) & \(1.82\pm{0.01}\) & \(3.07\pm{0.00}\) \\ \cline{2-5}
            & 3 & \(2.72\pm{0.01}\) & \(2.22\pm{0.01}\) & \(3.28\pm{0.02}\) \\ \cline{2-5}
            & 4 & \(3.04\pm{0.01}\) & \(2.79\pm{0.00}\) & \(3.70\pm{0.01}\) \\ \cline{2-5}
            & 5 & \(4.53\pm{0.02}\) & \(3.04\pm{0.00}\) & \(4.44\pm{0.03}\) \\ \cline{2-5}
            & 6 & \(4.58\pm{0.01}\) & \(3.51\pm{0.01}\) & \(4.66\pm{0.05}\) \\ \cline{2-5}
            & 7 & \(5.09\pm{0.00}\) & \(3.80\pm{0.01}\) & \(5.22\pm{0.00}\) \\ \cline{2-5}
            & 8 & \(5.41\pm{0.04}\) & \(4.35\pm{0.02}\) & \(5.74\pm{0.00}\) \\ \hline
        \end{tabular}
        \end{adjustbox}
    \end{subfigure}
    \hfill
    \begin{subfigure}{0.5\textwidth}
        \includegraphics[width=\textwidth]{bucle3=01-small}
    \end{subfigure}
    \caption{\underline{Tamaño pequeño}: Tiempos de ejecución vs nº de hilos}
    \label{fig:bucle3=01-small}
\end{figure}

%%% TABLA DE TIEMPOS E IMÁGENES %%%
\begin{figure}[H]
    \centering
    \begin{subfigure}{0.4\textwidth}
        \begin{adjustbox}{width=\textwidth} 
        \begin{tabular}{|c|c|c|c|c|}
            \hline
            \rowcolor{azul} \multicolumn{2}{|c|}{}&\multicolumn{3}{c|}{\textbf{Compiler}} \\ \hline
            \rowcolor{azul} \multicolumn{2}{|c|}{}&\texttt{clang}&\texttt{gcc}&\texttt{icc}\\ \hline
            \rowcolor{azul} \textbf{Testing size} & \textbf{Threads}&\multicolumn{3}{c|}{\textbf{Average time (s)}} \\ \hline
            \multirow{8}{2.5cm}{\textbf{02-medium}} & 1 & \(5.29\pm{0.01}\) & \(2.05\pm{0.06}\) & \(3.87\pm{0.04}\) \\ \cline{2-5}
            & 2 & \(6.35\pm{0.02}\) & \(3.59\pm{0.03}\) & \(6.36\pm{0.05}\) \\ \cline{2-5}
            & 3 & \(5.78\pm{0.00}\) & \(4.24\pm{0.02}\) & \(6.71\pm{0.03}\) \\ \cline{2-5}
            & 4 & \(6.21\pm{0.01}\) & \(5.30\pm{0.02}\) & \(7.54\pm{0.04}\) \\ \cline{2-5}
            & 5 & \(8.94\pm{0.05}\) & \(6.17\pm{0.00}\) & \(9.16\pm{0.02}\) \\ \cline{2-5}
            & 6 & \(9.56\pm{0.19}\) & \(6.69\pm{0.01}\) & \(9.43\pm{0.04}\) \\ \cline{2-5}
            & 7 & \(10.31\pm{0.06}\) & \(7.64\pm{0.00}\) & \(10.33\pm{0.00}\) \\ \cline{2-5}
            & 8 & \(11.27\pm{0.00}\) & \(8.66\pm{0.03}\) & \(11.21\pm{0.02}\) \\ \hline
        \end{tabular}
        \end{adjustbox}
    \end{subfigure}
    \hfill
    \begin{subfigure}{0.5\textwidth}
        \includegraphics[width=\textwidth]{bucle3=02-medium}
    \end{subfigure}
    \caption{\underline{Tamaño mediano}: Tiempos de ejecución vs nº de hilos}
    \label{bucle3=02-medium}
\end{figure}

%%% TABLA DE TIEMPOS E IMÁGENES %%%
\begin{figure}[H]
    \centering
    \begin{subfigure}{0.4\textwidth}
        \begin{adjustbox}{width=\textwidth} 
        \begin{tabular}{|c|c|c|c|c|}
            \hline
            \rowcolor{azul} \multicolumn{2}{|c|}{}&\multicolumn{3}{c|}{\textbf{Compiler}} \\ \hline
            \rowcolor{azul} \multicolumn{2}{|c|}{}&\texttt{clang}&\texttt{gcc}&\texttt{icc}\\ \hline
            \rowcolor{azul} \textbf{Testing size} & \textbf{Threads}&\multicolumn{3}{c|}{\textbf{Average time (s)}} \\ \hline
            \multirow{8}{1cm}{\textbf{03-large}} & 1 & \(8.62\pm{0.04}\) & \(3.03\pm{0.11}\) & \(6.27\pm{0.16}\) \\ \cline{2-5}
            & 2 & \(8.98\pm{0.06}\) & \(3.03\pm{0.09}\) & \(8.94\pm{0.08}\) \\ \cline{2-5}
            & 3 & \(8.15\pm{0.07}\) & \(4.77\pm{0.05}\) & \(9.04\pm{0.11}\) \\ \cline{2-5}
            & 4 & \(8.25\pm{0.03}\) & \(6.09\pm{0.04}\) & \(9.92\pm{0.00}\) \\ \cline{2-5}
            & 5 & \(11.96\pm{0.09}\) & \(7.25\pm{0.02}\) & \(12.09\pm{0.08}\) \\ \cline{2-5}
            & 6 & \(12.23\pm{0.09}\) & \(8.06\pm{0.02}\) & \(12.84\pm{0.04}\) \\ \cline{2-5}
            & 7 & \(13.77\pm{0.04}\) & \(9.78\pm{0.03}\) & \(13.59\pm{0.05}\) \\ \cline{2-5}
            & 8 & \(14.93\pm{0.04}\) & \(10.70\pm{0.03}\) & \(14.98\pm{0.08}\) \\ \hline
        \end{tabular}
        \end{adjustbox}
    \end{subfigure}
    \hfill
    \begin{subfigure}{0.5\textwidth}
        \includegraphics[width=\textwidth]{bucle3=03-large}
    \end{subfigure}
    \caption{\underline{Tamaño largo}: Tiempos de ejecución vs nº de hilos}
    \label{bucle3=03-large}
\end{figure}

\par Estos dos bucles anidados también podría ponerse el pragma \texttt{\#pragma omp parallel for collapse(2) reduction (max:difference)}
pero comprobando los tiempos se puede ver que empeora el tiempo de ejecución.


\par El primer y tercer bucle no se pueden vectorizar de forma automática porque tiene dependencias. No son inmediatamente
vectorizables, tampoco se puede reorganizar el código para evitar las dependencias ni crear instrucciones parciales para mitigar la
dependencia. No he encontrado ninguna transformación manual para poder vectorizarlos. Separar cada bucle en dos sería
ineficiente y el compilador descarta la vectorización de la instrucción que ya no tendría dependencia por tanto este cambio no
sería una solución. A lo mejor se puede programar usando los \texttt{instrinsics} vectoriales, que son un conjunto de funciones
disponibles para lenguajes de alto nivel que corresponden directamente a instrucciones vectoriales o hay alguna librería
optimizada que ya esté vectorizada, porque la idea de programar en lenguaje ensamblador hoy en día casi nunca se realiza.