\subsubsection{\textbf{5ºBucle:}}
\begin{listing}[firstnumber=35]
    @@ -36,3 +36,3 @@
    for (int i = 0; i < arrays_size; ++i) {
      total = total + C[i];
    }
\end{listing}
\par No hace falta analizar la ejecución desarrollada del bucle, porque mirando el código se puede decir a simple vista que hay una
\textbf{dependencia entre iteraciones}, puesto que en cada iteración se necesita obtener el valor total de la iteración anterior. Como este bucle
está siempre modificando la misma variable de la misma manera, el compilador puede vectorizar automáticamente este bucle
haciendo varias sumas parciales por cada iteración.