\subsubsection{\textbf{2ºBucle:}}
\begin{listing}[firstnumber=21]
    @@ -22,4 +22,4 @@
    for (int i = 0; i < arrays_size - 1; ++i) {
      A[i] = A[i] - B[i];
      B[i + 1] = C[i] * 2;
    }
\end{listing}
\par Analizando la ejecución desarrollada del bucle se pueden ver las dependencias:
\begin{listing}[numbers=none]
    // Dentro del bucle
    A[0] = A[0] - B[0];
    B[1] = C[0] * 2;

    A[1] = A[1] - B[1];
    B[2] = C[1] * 2;

            ...

    A[arrays_size-1] = A[arrays_size-1] - B[arrays_size-1];
    B[arrays_size] = C[arrays_size-1] * 2; 
\end{listing}
\par En este bucle se producen dependencias \textbf{RAW} porque en la línea 23 se quiere calcular \texttt{A[i]}, pero depende del valor \texttt{B[i]} obtenido
en la anterior iteración. Este caso es mucho más sencillo, la dependencia no afectaría a la vectorización porque el compilador lo
resolvería reordenando las instrucciones, internamente en código ensamblador, el bucle se transformaría en el siguiente código
para eliminar la dependencia.
\begin{listing}[numbers=none]
    // Fuera del bucle
    A[0] = A[0] - B[0];
    // Dentro del bucle
    B[1] = C[0] * 2;
    A[1] = A[1] - B[1];

    B[2] = C[1] * 2;
    A[2] = A[2] - B[2];

            ...
    B[arrays_size-1] = C[arrays_size-2] * 2;
    A[arrays_size-1] = A[arrays_size-1] - B[arrays_size-1];
    //Fuera del bucle
    B[arrays_size] = C[arrays_size-1] * 2; 
\end{listing}
\begin{listing}[firstnumber=21]
    @@ -22,4 +22,6 @@
  + A[0] = A[0] - B[0];
  + for (int i = 0; i < arrays_size - 2; ++i) {
  +    B[i + 1] = C[i] * 2;
  +    A[i] = A[i] - B[i];
  + }
  + B[arrays_size] = C[arrays_size-1] * 2;
\end{listing}