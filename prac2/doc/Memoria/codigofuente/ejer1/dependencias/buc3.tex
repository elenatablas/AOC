\subsubsection{\textbf{3ºBucle:}}
\begin{listing}[firstnumber=26]
    @@ -27,4 +27,4 @@
    for (int i = arrays_size; i > 0; --i) {
      A[i - 1] = A[i] + C[i];
      B[i] = C[i] - B[i];
    }
\end{listing}
\par Analizando la ejecución desarrollada del bucle se pueden ver las dependencias:
\newpage
\begin{listing}[numbers=none]
    A[array_size-1] = A[array_size] + C[array_size];
    B[array_size] = C[array_size] - B[array_size];

    A[array_size-2] = A[array_size-1] + C[array_size-1];
    B[array_size-1] = C[array_size-1] - B[array_size-1];

    A[array_size-3] = A[array_size-2] + C[array_size-2];
    B[array_size-2] = C[array_size-2] - B[array_size-2];

                                ...

    A[0] = A[1] + C[1];
    B[1] = A[1] - B[1];
\end{listing}
\par Es parecido al primer bucle, existe una \textbf{dependencia entre iteraciones} de tal forma que para obtener el valor de \texttt{A[i-1]} se necesita
la ejecución secuencial de la anterior iteración que obtiene el valor de \texttt{A[i]}. La otra dependencia dentro de la misma iteración sería
ineficiente ejecutarlas en paralelo, por tanto, el compilador no podrá vectorizar automáticamente.