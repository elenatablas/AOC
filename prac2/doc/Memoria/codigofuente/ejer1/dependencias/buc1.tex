\subsubsection{\textbf{1ºBucle:}}
\begin{listing}[firstnumber=16]
    @@ -17,4 +17,4 @@
    for (int i = 0; i < arrays_size - 1; ++i) {
      A[i] = i * 2 / 3;
      B[i + 1] = A[i] - B[i];
    }
\end{listing}
\par Analizando la ejecución desarrollada del bucle se pueden ver las dependencias:
\newpage
\begin{listing}[numbers=none]
    A[0] = 0 * 2 / 3;
    B[1] = A[0] - B[0];

    A[1] = 1* 2 / 3;
    B[2] = A[1] - B[1];

    A[2] = 2 * 2 / 3;
    B[3] = A[2] - B[2];

            ...

    A[arrays_size-1] = 2 * 2 / 3;
    B[arrays_size] = A[arrays_size-1] - B[arrays_size-1];
\end{listing}
\par Dentro de una misma iteración hay una \textbf{dependencia RAW} donde el valor de \texttt{A[i]} se utiliza en la siguiente instrucción para
obtener el valor de \texttt{B[i+1]}. También, se produce una \textbf{dependencia RAW} entre iteraciones, porque para obtener el valor de \texttt{B[i+1]}
hace falta el valor de \texttt{B[i]} que se ha calculado en la anterior iteración.
\par La primera dependencia no afectaría a una posible vectorización automática, porque se puede evitar sustituyendo en la línea 19
\texttt{A[i]} por las operaciones que hace para poder ejecutarlas en paralelo o dejar el código igual para que se ejecuten secuencialmente
aunque este cambio no mejora mucho el código, sería ineficiente. En cambio, el compilador no puede ejecutar en paralelo las
iteraciones del bucle, porque la segunda dependencia impide la vectorización automática.