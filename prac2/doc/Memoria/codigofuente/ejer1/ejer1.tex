%%%%%%%%%%%%%%%%%%%%%% Procesadores CMP: usando la vectorización y la paralelización - EJERCICIO 1 %%%%%%%%%%%%%%%%%%%%%%
\section{Ejercicio 1}
\subsection{Enunciado}
\begin{ejer}
    \textbf{1.-} En el subdirectorio \texttt{vectorización} hay un pequeño programa que tiene 6 bucles.
    \begin{itemize}
        \item Para cada uno de los bucles del programa, identifica las dependencias entre iteraciones que tiene
    y explica cómo afectarían a una posible vectorización automática.
        \item Utiliza el \texttt{Makefile} incluido para compilar el programa con el compilador \texttt{ICC} y generar
    el informe de optimización. Para cada bucle mencionado en el informe, explica por qué ha sido vectorizado o por qué no 
    se ha podido vectorizar. En especial, para los bucles en los que se hubieran identificado dependencias entre iteraciones
     en el punto anterior:
        \begin{itemize}
            \item Explica en caso de que haya sido vectorizado qué transformaciones ha aplicado automáticamente el compilador.
            \item Y en caso de que no haya sido vectorizado, indica si crees que existe alguna transformación manual posible para
            que se pudiera vectorizar.
        \end{itemize}
    \end{itemize}
\end{ejer}

\subsection{Identificación dependencias}
\par Solo analizo los bucles internos, porque la vectorización consiste en ejecutar en paralelo las operaciones matemáticas realizadas en 
los bucles internos de aplicaciones de código científico.
\par Hay varias estrategias para usar código vectorial en los programas, en este caso dejo que el compilador 
vectorice automáticamente.

\subsubsection{\textbf{1ºBucle:}}
\begin{listing}
    @@ -20,3 +20,4 @@
\end{listing}


\subsubsection{\textbf{2ºBucle:}}
\begin{listing}[firstnumber=21]
    @@ -22,4 +22,4 @@
    for (int i = 0; i < arrays_size - 1; ++i) {
      A[i] = A[i] - B[i];
      B[i + 1] = C[i] * 2;
    }
\end{listing}
\par Analizando la ejecución desarrollada del bucle se pueden ver las dependencias:
\begin{listing}[numbers=none]
    // Dentro del bucle
    A[0] = A[0] - B[0];
    B[1] = C[0] * 2;

    A[1] = A[1] - B[1];
    B[2] = C[1] * 2;

            ...

    A[arrays_size-1] = A[arrays_size-1] - B[arrays_size-1];
    B[arrays_size] = C[arrays_size-1] * 2; 
\end{listing}
\par En este bucle se producen dependencias \textbf{RAW} porque en la línea 23 se quiere calcular \texttt{A[i]}, pero depende del valor \texttt{B[i]} obtenido
en la anterior iteración. Este caso es mucho más sencillo, la dependencia no afectaría a la vectorización porque el compilador lo
resolvería reordenando las instrucciones, internamente en código ensamblador, el bucle se transformaría en el siguiente código
para eliminar la dependencia.
\begin{listing}[numbers=none]
    // Fuera del bucle
    A[0] = A[0] - B[0];
    // Dentro del bucle
    B[1] = C[0] * 2;
    A[1] = A[1] - B[1];

    B[2] = C[1] * 2;
    A[2] = A[2] - B[2];

            ...
    B[arrays_size-1] = C[arrays_size-2] * 2;
    A[arrays_size-1] = A[arrays_size-1] - B[arrays_size-1];
    //Fuera del bucle
    B[arrays_size] = C[arrays_size-1] * 2; 
\end{listing}
\begin{listing}[firstnumber=21]
    @@ -22,4 +22,6 @@
    + A[0] = A[0] - B[0];
    + for (int i = 0; i < arrays_size - 2; ++i) {
    +    B[i + 1] = C[i] * 2;
    +    A[i] = A[i] - B[i];
    }
    + B[arrays_size] = C[arrays_size-1] * 2;
\end{listing}

\subsubsection{\textbf{3ºBucle:}}

\par El tercer bucle es el bucle interior que está anidado \texttt{for (size\_t j = 1; j < state.width - 1; ++j)},
 se añadiría el mismo \texttt{pragma} que en el caso anterior pero el rendimiento es mucho más lento.

%%% TABLA DE TIEMPOS E IMÁGENES %%%
\begin{figure}[H]
    \centering
    \begin{subfigure}{0.4\textwidth}
        \begin{adjustbox}{width=\textwidth} 
        \begin{tabular}{|c|c|c|c|c|}
            \hline
            \rowcolor{azul} \multicolumn{2}{|c|}{}&\multicolumn{3}{c|}{\textbf{Compiler}} \\ \hline
            \rowcolor{azul} \multicolumn{2}{|c|}{}&\texttt{clang}&\texttt{gcc}&\texttt{icc}\\ \hline
            \rowcolor{azul} \textbf{Testing size} & \textbf{Threads}&\multicolumn{3}{c|}{\textbf{Average time (s)}} \\ \hline
            \multirow{8}{1cm}{\textbf{01-small}} & 1 & \(1.98\pm{0.00}\) & \(1.30\pm{0.14}\) & \(1.52\pm{0.01}\) \\ \cline{2-5}
            & 2 & \(2.78\pm{0.03}\) & \(1.82\pm{0.01}\) & \(3.07\pm{0.00}\) \\ \cline{2-5}
            & 3 & \(2.72\pm{0.01}\) & \(2.22\pm{0.01}\) & \(3.28\pm{0.02}\) \\ \cline{2-5}
            & 4 & \(3.04\pm{0.01}\) & \(2.79\pm{0.00}\) & \(3.70\pm{0.01}\) \\ \cline{2-5}
            & 5 & \(4.53\pm{0.02}\) & \(3.04\pm{0.00}\) & \(4.44\pm{0.03}\) \\ \cline{2-5}
            & 6 & \(4.58\pm{0.01}\) & \(3.51\pm{0.01}\) & \(4.66\pm{0.05}\) \\ \cline{2-5}
            & 7 & \(5.09\pm{0.00}\) & \(3.80\pm{0.01}\) & \(5.22\pm{0.00}\) \\ \cline{2-5}
            & 8 & \(5.41\pm{0.04}\) & \(4.35\pm{0.02}\) & \(5.74\pm{0.00}\) \\ \hline
        \end{tabular}
        \end{adjustbox}
    \end{subfigure}
    \hfill
    \begin{subfigure}{0.5\textwidth}
        \includegraphics[width=\textwidth]{bucle3=01-small}
    \end{subfigure}
    \caption{\underline{Tamaño pequeño}: Tiempos de ejecución vs nº de hilos}
    \label{fig:bucle3=01-small}
\end{figure}

%%% TABLA DE TIEMPOS E IMÁGENES %%%
\begin{figure}[H]
    \centering
    \begin{subfigure}{0.4\textwidth}
        \begin{adjustbox}{width=\textwidth} 
        \begin{tabular}{|c|c|c|c|c|}
            \hline
            \rowcolor{azul} \multicolumn{2}{|c|}{}&\multicolumn{3}{c|}{\textbf{Compiler}} \\ \hline
            \rowcolor{azul} \multicolumn{2}{|c|}{}&\texttt{clang}&\texttt{gcc}&\texttt{icc}\\ \hline
            \rowcolor{azul} \textbf{Testing size} & \textbf{Threads}&\multicolumn{3}{c|}{\textbf{Average time (s)}} \\ \hline
            \multirow{8}{2.5cm}{\textbf{02-medium}} & 1 & \(5.29\pm{0.01}\) & \(2.05\pm{0.06}\) & \(3.87\pm{0.04}\) \\ \cline{2-5}
            & 2 & \(6.35\pm{0.02}\) & \(3.59\pm{0.03}\) & \(6.36\pm{0.05}\) \\ \cline{2-5}
            & 3 & \(5.78\pm{0.00}\) & \(4.24\pm{0.02}\) & \(6.71\pm{0.03}\) \\ \cline{2-5}
            & 4 & \(6.21\pm{0.01}\) & \(5.30\pm{0.02}\) & \(7.54\pm{0.04}\) \\ \cline{2-5}
            & 5 & \(8.94\pm{0.05}\) & \(6.17\pm{0.00}\) & \(9.16\pm{0.02}\) \\ \cline{2-5}
            & 6 & \(9.56\pm{0.19}\) & \(6.69\pm{0.01}\) & \(9.43\pm{0.04}\) \\ \cline{2-5}
            & 7 & \(10.31\pm{0.06}\) & \(7.64\pm{0.00}\) & \(10.33\pm{0.00}\) \\ \cline{2-5}
            & 8 & \(11.27\pm{0.00}\) & \(8.66\pm{0.03}\) & \(11.21\pm{0.02}\) \\ \hline
        \end{tabular}
        \end{adjustbox}
    \end{subfigure}
    \hfill
    \begin{subfigure}{0.5\textwidth}
        \includegraphics[width=\textwidth]{bucle3=02-medium}
    \end{subfigure}
    \caption{\underline{Tamaño mediano}: Tiempos de ejecución vs nº de hilos}
    \label{bucle3=02-medium}
\end{figure}

%%% TABLA DE TIEMPOS E IMÁGENES %%%
\begin{figure}[H]
    \centering
    \begin{subfigure}{0.4\textwidth}
        \begin{adjustbox}{width=\textwidth} 
        \begin{tabular}{|c|c|c|c|c|}
            \hline
            \rowcolor{azul} \multicolumn{2}{|c|}{}&\multicolumn{3}{c|}{\textbf{Compiler}} \\ \hline
            \rowcolor{azul} \multicolumn{2}{|c|}{}&\texttt{clang}&\texttt{gcc}&\texttt{icc}\\ \hline
            \rowcolor{azul} \textbf{Testing size} & \textbf{Threads}&\multicolumn{3}{c|}{\textbf{Average time (s)}} \\ \hline
            \multirow{8}{1cm}{\textbf{03-large}} & 1 & \(8.62\pm{0.04}\) & \(3.03\pm{0.11}\) & \(6.27\pm{0.16}\) \\ \cline{2-5}
            & 2 & \(8.98\pm{0.06}\) & \(3.03\pm{0.09}\) & \(8.94\pm{0.08}\) \\ \cline{2-5}
            & 3 & \(8.15\pm{0.07}\) & \(4.77\pm{0.05}\) & \(9.04\pm{0.11}\) \\ \cline{2-5}
            & 4 & \(8.25\pm{0.03}\) & \(6.09\pm{0.04}\) & \(9.92\pm{0.00}\) \\ \cline{2-5}
            & 5 & \(11.96\pm{0.09}\) & \(7.25\pm{0.02}\) & \(12.09\pm{0.08}\) \\ \cline{2-5}
            & 6 & \(12.23\pm{0.09}\) & \(8.06\pm{0.02}\) & \(12.84\pm{0.04}\) \\ \cline{2-5}
            & 7 & \(13.77\pm{0.04}\) & \(9.78\pm{0.03}\) & \(13.59\pm{0.05}\) \\ \cline{2-5}
            & 8 & \(14.93\pm{0.04}\) & \(10.70\pm{0.03}\) & \(14.98\pm{0.08}\) \\ \hline
        \end{tabular}
        \end{adjustbox}
    \end{subfigure}
    \hfill
    \begin{subfigure}{0.5\textwidth}
        \includegraphics[width=\textwidth]{bucle3=03-large}
    \end{subfigure}
    \caption{\underline{Tamaño largo}: Tiempos de ejecución vs nº de hilos}
    \label{bucle3=03-large}
\end{figure}

\par Estos dos bucles anidados también podría ponerse el pragma \texttt{\#pragma omp parallel for collapse(2) reduction (max:difference)}
pero comprobando los tiempos se puede ver que empeora el tiempo de ejecución.

\subsubsection{\textbf{4ºBucle:}}
\begin{listing}[firstnumber=108]
    @@ -109,3 +109,3 @@
    LOOP BEGIN at loops.cpp(32,5)
        remark #15300: LOOP WAS VECTORIZED
    LOOP END
\end{listing}
\par Ha sido vectorizado porque no hay dependencias, el tamaño y la forma de los arrays \texttt{A, B y C} lo conoce porque está
directamente visible en \texttt{double A[arrays\_size];} 
\texttt{double B[arrays\_size];} \texttt{double C[arrays\_size];}
\subsubsection{\textbf{5ºBucle:}}
\begin{listing}[firstnumber=126]
    @@ -127,3 +127,3 @@
    LOOP BEGIN at loops.cpp(36,5)
        remark #15300: LOOP WAS VECTORIZED
    LOOP END
\end{listing}
\par Para saber como optimiza este bucle podemos ver el código ensamblador generado con el comando \texttt{\$ objdump -S loops | less}
en el que se puede ver como utiliza 8 registros para almacenar los totales parciales de ocho en ocho en cada iteración.

\begin{listing}[numbers=none, basicstyle=\scriptsize\ttfamily]
    400ee5: 0f ae f0                mfence
    400ee8: c5 7d 6f d1             vmovdqa %ymm1,%ymm10
    400eec: 33 c0                   xor %eax,%eax
    400eee: c4 41 25 57 db          vxorpd %ymm11,%ymm11,%ymm11
    400ef3: c5 7d 6f c9             vmovdqa %ymm1,%ymm9
    400ef7: c5 7d 6f c1             vmovdqa %ymm1,%ymm8
    400efb: c5 fd 6f f9             vmovdqa %ymm1,%ymm7
    400eff: c5 fd 6f f1             vmovdqa %ymm1,%ymm6
    400f03: c5 fd 28 e9             vmovapd %ymm1,%ymm5
    400f07: c5 fd 28 e1             vmovapd %ymm1,%ymm4
    400f0b: 0f 1f 44 00 00          nopl 0x0(%rax,%rax,1)
    400f10: c5 25 58 1c c5 60 61    vaddpd 0x1546160(,%rax,8),%ymm11,%ymm11
    400f17: 54 01
    400f19: c5 2d 58 14 c5 80 61    vaddpd 0x1546180(,%rax,8),%ymm10,%ymm10
    400f20: 54 01
    400f22: c5 35 58 0c c5 a0 61    vaddpd 0x15461a0(,%rax,8),%ymm9,%ymm9
    400f29: 54 01
    400f2b: c5 3d 58 04 c5 c0 61    vaddpd 0x15461c0(,%rax,8),%ymm8,%ymm8
    400f32: 54 01
    400f34: c5 c5 58 3c c5 e0 61    vaddpd 0x15461e0(,%rax,8),%ymm7,%ymm7
    400f3b: 54 01
    400f3d: c5 cd 58 34 c5 00 62    vaddpd 0x1546200(,%rax,8),%ymm6,%ymm6
    400f44: 54 01
    400f46: c5 d5 58 2c c5 20 62    vaddpd 0x1546220(,%rax,8),%ymm5,%ymm5
    400f4d: 54 01
    400f4f: c5 dd 58 24 c5 40 62    vaddpd 0x1546240(,%rax,8),%ymm4,%ymm4
    400f56: 54 01
    400f58: 48 83 c0 20             add $0x20,%rax
    400f5c: 48 3d 40 42 0f 00       cmp $0xf4240,%rax
    400f62: 72 ac                   jb 400f10 <main+0x200>
    400f64: c4 41 25 58 d2          vaddpd %ymm10,%ymm11,%ymm10
    400f69: fe c1                   inc %cl
    400f6b: c4 41 35 58 c0          vaddpd %ymm8,%ymm9,%ymm8
    400f70: c5 c5 58 f6             vaddpd %ymm6,%ymm7,%ymm6
    400f74: c5 d5 58 e4             vaddpd %ymm4,%ymm5,%ymm4
    400f78: c4 41 2d 58 c8          vaddpd %ymm8,%ymm10,%ymm9
    400f7d: c5 cd 58 ec             vaddpd %ymm4,%ymm6,%ymm5
    400f81: c5 b5 58 fd             vaddpd %ymm5,%ymm9,%ymm7
    400f85: c4 c3 7d 19 fb 01       vextractf128 $0x1,%ymm7,%xmm11
    400f8b: c4 41 41 58 e3          vaddpd %xmm11,%xmm7,%xmm12
    400f90: c4 41 19 15 ec          vunpckhpd %xmm12,%xmm12,%xmm13
    400f95: c4 41 1b 58 f5          vaddsd %xmm13,%xmm12,%xmm14
    400f9a: c5 8b 58 c0             vaddsd %xmm0,%xmm14,%xmm0
    400f9e: 80 f9 64                cmp $0x64,%cl
    400fa1: 0f 82 b2 fd ff ff       jb 400d59 <main+0x49>
\end{listing}
\par En \texttt{C} el código equivalente sería:
\begin{listing}[firstnumber=35]
    @@ -36,3 +36,11 @@
    + for (int i = 0; i < arrays_size; i+=8)
    + {
    +    ymm11 = ymm11 + c[i];
    +    ymm10 = ymm10 + c[i+1];
    +    ymm9 = ymm9 + c[i+2];
    +    ymm8 = ymm8 + c[i+3];
    +    ymm7 = ymm7 + c[i+4];
    +    ymm6 = ymm6 + c[i+5];
    +    ymm5 = ymm5 + c[i+6];
    +    ymm4 = ymm4 + c[i+7];
    + }
    + total = ymm11+ymm10+ymm9+ymm8+ymm7+ymm6+ymm5+ymm4+total;
\end{listing}
\subsection{Vectorización automática}
\par La vectorización es automática, hay que darle al compilador suficiente información para que pueda tomar las decisiones correctas
y pueda optimizar cuando sea vectorizable el programa, para ello es necesario comprobar que al compilar se utiliza la opción de
optimización \texttt{-O2} o superior.

\par Nos indica los parámetros que se han utilizado: \texttt{Report from: Loop nest, Vector \& Auto-parallelization optimizations
[loop, vec, par]}
\begin{itemize}
    \item \textbf{prac2-ejercicios/vectorización/loops-icc.optrpt}

\begin{listing}[firstnumber=44]
    @@ -45,4 +45,4 @@ 
    LOOP BEGIN at loops.cpp(15,3)
        remark #15542: loop was not vectorized: inner loop was already vectorized
        ...
    LOOP END
\end{listing}
\end{itemize}
\par El bucle exterior no se vectoriza, si no los bucles que están dentro, pero como es un bucle está representado en el informe.

\subsubsection{\textbf{2ºBucle:}}
\begin{listing}[firstnumber=21]
    @@ -22,4 +22,4 @@
    for (int i = 0; i < arrays_size - 1; ++i) {
      A[i] = A[i] - B[i];
      B[i + 1] = C[i] * 2;
    }
\end{listing}
\par Analizando la ejecución desarrollada del bucle se pueden ver las dependencias:
\begin{listing}[numbers=none]
    // Dentro del bucle
    A[0] = A[0] - B[0];
    B[1] = C[0] * 2;

    A[1] = A[1] - B[1];
    B[2] = C[1] * 2;

            ...

    A[arrays_size-1] = A[arrays_size-1] - B[arrays_size-1];
    B[arrays_size] = C[arrays_size-1] * 2; 
\end{listing}
\par En este bucle se producen dependencias \textbf{RAW} porque en la línea 23 se quiere calcular \texttt{A[i]}, pero depende del valor \texttt{B[i]} obtenido
en la anterior iteración. Este caso es mucho más sencillo, la dependencia no afectaría a la vectorización porque el compilador lo
resolvería reordenando las instrucciones, internamente en código ensamblador, el bucle se transformaría en el siguiente código
para eliminar la dependencia.
\begin{listing}[numbers=none]
    // Fuera del bucle
    A[0] = A[0] - B[0];
    // Dentro del bucle
    B[1] = C[0] * 2;
    A[1] = A[1] - B[1];

    B[2] = C[1] * 2;
    A[2] = A[2] - B[2];

            ...
    B[arrays_size-1] = C[arrays_size-2] * 2;
    A[arrays_size-1] = A[arrays_size-1] - B[arrays_size-1];
    //Fuera del bucle
    B[arrays_size] = C[arrays_size-1] * 2; 
\end{listing}
\begin{listing}[firstnumber=21]
    @@ -22,4 +22,6 @@
    + A[0] = A[0] - B[0];
    + for (int i = 0; i < arrays_size - 2; ++i) {
    +    B[i + 1] = C[i] * 2;
    +    A[i] = A[i] - B[i];
    }
    + B[arrays_size] = C[arrays_size-1] * 2;
\end{listing}
\subsubsection{\textbf{4ºBucle:}}
\begin{listing}[firstnumber=108]
    @@ -109,3 +109,3 @@
    LOOP BEGIN at loops.cpp(32,5)
        remark #15300: LOOP WAS VECTORIZED
    LOOP END
\end{listing}
\par Ha sido vectorizado porque no hay dependencias, el tamaño y la forma de los arrays \texttt{A, B y C} lo conoce porque está
directamente visible en \texttt{double A[arrays\_size];} 
\texttt{double B[arrays\_size];} \texttt{double C[arrays\_size];}
\subsubsection{\textbf{5ºBucle:}}
\begin{listing}[firstnumber=126]
    @@ -127,3 +127,3 @@
    LOOP BEGIN at loops.cpp(36,5)
        remark #15300: LOOP WAS VECTORIZED
    LOOP END
\end{listing}
\par Para saber como optimiza este bucle podemos ver el código ensamblador generado con el comando \texttt{\$ objdump -S loops | less}
en el que se puede ver como utiliza 8 registros para almacenar los totales parciales de ocho en ocho en cada iteración.

\begin{listing}[numbers=none, basicstyle=\scriptsize\ttfamily]
    400ee5: 0f ae f0                mfence
    400ee8: c5 7d 6f d1             vmovdqa %ymm1,%ymm10
    400eec: 33 c0                   xor %eax,%eax
    400eee: c4 41 25 57 db          vxorpd %ymm11,%ymm11,%ymm11
    400ef3: c5 7d 6f c9             vmovdqa %ymm1,%ymm9
    400ef7: c5 7d 6f c1             vmovdqa %ymm1,%ymm8
    400efb: c5 fd 6f f9             vmovdqa %ymm1,%ymm7
    400eff: c5 fd 6f f1             vmovdqa %ymm1,%ymm6
    400f03: c5 fd 28 e9             vmovapd %ymm1,%ymm5
    400f07: c5 fd 28 e1             vmovapd %ymm1,%ymm4
    400f0b: 0f 1f 44 00 00          nopl 0x0(%rax,%rax,1)
    400f10: c5 25 58 1c c5 60 61    vaddpd 0x1546160(,%rax,8),%ymm11,%ymm11
    400f17: 54 01
    400f19: c5 2d 58 14 c5 80 61    vaddpd 0x1546180(,%rax,8),%ymm10,%ymm10
    400f20: 54 01
    400f22: c5 35 58 0c c5 a0 61    vaddpd 0x15461a0(,%rax,8),%ymm9,%ymm9
    400f29: 54 01
    400f2b: c5 3d 58 04 c5 c0 61    vaddpd 0x15461c0(,%rax,8),%ymm8,%ymm8
    400f32: 54 01
    400f34: c5 c5 58 3c c5 e0 61    vaddpd 0x15461e0(,%rax,8),%ymm7,%ymm7
    400f3b: 54 01
    400f3d: c5 cd 58 34 c5 00 62    vaddpd 0x1546200(,%rax,8),%ymm6,%ymm6
    400f44: 54 01
    400f46: c5 d5 58 2c c5 20 62    vaddpd 0x1546220(,%rax,8),%ymm5,%ymm5
    400f4d: 54 01
    400f4f: c5 dd 58 24 c5 40 62    vaddpd 0x1546240(,%rax,8),%ymm4,%ymm4
    400f56: 54 01
    400f58: 48 83 c0 20             add $0x20,%rax
    400f5c: 48 3d 40 42 0f 00       cmp $0xf4240,%rax
    400f62: 72 ac                   jb 400f10 <main+0x200>
    400f64: c4 41 25 58 d2          vaddpd %ymm10,%ymm11,%ymm10
    400f69: fe c1                   inc %cl
    400f6b: c4 41 35 58 c0          vaddpd %ymm8,%ymm9,%ymm8
    400f70: c5 c5 58 f6             vaddpd %ymm6,%ymm7,%ymm6
    400f74: c5 d5 58 e4             vaddpd %ymm4,%ymm5,%ymm4
    400f78: c4 41 2d 58 c8          vaddpd %ymm8,%ymm10,%ymm9
    400f7d: c5 cd 58 ec             vaddpd %ymm4,%ymm6,%ymm5
    400f81: c5 b5 58 fd             vaddpd %ymm5,%ymm9,%ymm7
    400f85: c4 c3 7d 19 fb 01       vextractf128 $0x1,%ymm7,%xmm11
    400f8b: c4 41 41 58 e3          vaddpd %xmm11,%xmm7,%xmm12
    400f90: c4 41 19 15 ec          vunpckhpd %xmm12,%xmm12,%xmm13
    400f95: c4 41 1b 58 f5          vaddsd %xmm13,%xmm12,%xmm14
    400f9a: c5 8b 58 c0             vaddsd %xmm0,%xmm14,%xmm0
    400f9e: 80 f9 64                cmp $0x64,%cl
    400fa1: 0f 82 b2 fd ff ff       jb 400d59 <main+0x49>
\end{listing}
\par En \texttt{C} el código equivalente sería:
\begin{listing}[firstnumber=35]
    @@ -36,3 +36,11 @@
    + for (int i = 0; i < arrays_size; i+=8)
    + {
    +    ymm11 = ymm11 + c[i];
    +    ymm10 = ymm10 + c[i+1];
    +    ymm9 = ymm9 + c[i+2];
    +    ymm8 = ymm8 + c[i+3];
    +    ymm7 = ymm7 + c[i+4];
    +    ymm6 = ymm6 + c[i+5];
    +    ymm5 = ymm5 + c[i+6];
    +    ymm4 = ymm4 + c[i+7];
    + }
    + total = ymm11+ymm10+ymm9+ymm8+ymm7+ymm6+ymm5+ymm4+total;
\end{listing}
\subsection{Transformación manual para vectorizar}

\subsubsection{\textbf{1ºBucle:}}
\begin{listing}
    @@ -20,3 +20,4 @@
\end{listing}


\newpage

\subsubsection{\textbf{3ºBucle:}}

\par El tercer bucle es el bucle interior que está anidado \texttt{for (size\_t j = 1; j < state.width - 1; ++j)},
 se añadiría el mismo \texttt{pragma} que en el caso anterior pero el rendimiento es mucho más lento.

%%% TABLA DE TIEMPOS E IMÁGENES %%%
\begin{figure}[H]
    \centering
    \begin{subfigure}{0.4\textwidth}
        \begin{adjustbox}{width=\textwidth} 
        \begin{tabular}{|c|c|c|c|c|}
            \hline
            \rowcolor{azul} \multicolumn{2}{|c|}{}&\multicolumn{3}{c|}{\textbf{Compiler}} \\ \hline
            \rowcolor{azul} \multicolumn{2}{|c|}{}&\texttt{clang}&\texttt{gcc}&\texttt{icc}\\ \hline
            \rowcolor{azul} \textbf{Testing size} & \textbf{Threads}&\multicolumn{3}{c|}{\textbf{Average time (s)}} \\ \hline
            \multirow{8}{1cm}{\textbf{01-small}} & 1 & \(1.98\pm{0.00}\) & \(1.30\pm{0.14}\) & \(1.52\pm{0.01}\) \\ \cline{2-5}
            & 2 & \(2.78\pm{0.03}\) & \(1.82\pm{0.01}\) & \(3.07\pm{0.00}\) \\ \cline{2-5}
            & 3 & \(2.72\pm{0.01}\) & \(2.22\pm{0.01}\) & \(3.28\pm{0.02}\) \\ \cline{2-5}
            & 4 & \(3.04\pm{0.01}\) & \(2.79\pm{0.00}\) & \(3.70\pm{0.01}\) \\ \cline{2-5}
            & 5 & \(4.53\pm{0.02}\) & \(3.04\pm{0.00}\) & \(4.44\pm{0.03}\) \\ \cline{2-5}
            & 6 & \(4.58\pm{0.01}\) & \(3.51\pm{0.01}\) & \(4.66\pm{0.05}\) \\ \cline{2-5}
            & 7 & \(5.09\pm{0.00}\) & \(3.80\pm{0.01}\) & \(5.22\pm{0.00}\) \\ \cline{2-5}
            & 8 & \(5.41\pm{0.04}\) & \(4.35\pm{0.02}\) & \(5.74\pm{0.00}\) \\ \hline
        \end{tabular}
        \end{adjustbox}
    \end{subfigure}
    \hfill
    \begin{subfigure}{0.5\textwidth}
        \includegraphics[width=\textwidth]{bucle3=01-small}
    \end{subfigure}
    \caption{\underline{Tamaño pequeño}: Tiempos de ejecución vs nº de hilos}
    \label{fig:bucle3=01-small}
\end{figure}

%%% TABLA DE TIEMPOS E IMÁGENES %%%
\begin{figure}[H]
    \centering
    \begin{subfigure}{0.4\textwidth}
        \begin{adjustbox}{width=\textwidth} 
        \begin{tabular}{|c|c|c|c|c|}
            \hline
            \rowcolor{azul} \multicolumn{2}{|c|}{}&\multicolumn{3}{c|}{\textbf{Compiler}} \\ \hline
            \rowcolor{azul} \multicolumn{2}{|c|}{}&\texttt{clang}&\texttt{gcc}&\texttt{icc}\\ \hline
            \rowcolor{azul} \textbf{Testing size} & \textbf{Threads}&\multicolumn{3}{c|}{\textbf{Average time (s)}} \\ \hline
            \multirow{8}{2.5cm}{\textbf{02-medium}} & 1 & \(5.29\pm{0.01}\) & \(2.05\pm{0.06}\) & \(3.87\pm{0.04}\) \\ \cline{2-5}
            & 2 & \(6.35\pm{0.02}\) & \(3.59\pm{0.03}\) & \(6.36\pm{0.05}\) \\ \cline{2-5}
            & 3 & \(5.78\pm{0.00}\) & \(4.24\pm{0.02}\) & \(6.71\pm{0.03}\) \\ \cline{2-5}
            & 4 & \(6.21\pm{0.01}\) & \(5.30\pm{0.02}\) & \(7.54\pm{0.04}\) \\ \cline{2-5}
            & 5 & \(8.94\pm{0.05}\) & \(6.17\pm{0.00}\) & \(9.16\pm{0.02}\) \\ \cline{2-5}
            & 6 & \(9.56\pm{0.19}\) & \(6.69\pm{0.01}\) & \(9.43\pm{0.04}\) \\ \cline{2-5}
            & 7 & \(10.31\pm{0.06}\) & \(7.64\pm{0.00}\) & \(10.33\pm{0.00}\) \\ \cline{2-5}
            & 8 & \(11.27\pm{0.00}\) & \(8.66\pm{0.03}\) & \(11.21\pm{0.02}\) \\ \hline
        \end{tabular}
        \end{adjustbox}
    \end{subfigure}
    \hfill
    \begin{subfigure}{0.5\textwidth}
        \includegraphics[width=\textwidth]{bucle3=02-medium}
    \end{subfigure}
    \caption{\underline{Tamaño mediano}: Tiempos de ejecución vs nº de hilos}
    \label{bucle3=02-medium}
\end{figure}

%%% TABLA DE TIEMPOS E IMÁGENES %%%
\begin{figure}[H]
    \centering
    \begin{subfigure}{0.4\textwidth}
        \begin{adjustbox}{width=\textwidth} 
        \begin{tabular}{|c|c|c|c|c|}
            \hline
            \rowcolor{azul} \multicolumn{2}{|c|}{}&\multicolumn{3}{c|}{\textbf{Compiler}} \\ \hline
            \rowcolor{azul} \multicolumn{2}{|c|}{}&\texttt{clang}&\texttt{gcc}&\texttt{icc}\\ \hline
            \rowcolor{azul} \textbf{Testing size} & \textbf{Threads}&\multicolumn{3}{c|}{\textbf{Average time (s)}} \\ \hline
            \multirow{8}{1cm}{\textbf{03-large}} & 1 & \(8.62\pm{0.04}\) & \(3.03\pm{0.11}\) & \(6.27\pm{0.16}\) \\ \cline{2-5}
            & 2 & \(8.98\pm{0.06}\) & \(3.03\pm{0.09}\) & \(8.94\pm{0.08}\) \\ \cline{2-5}
            & 3 & \(8.15\pm{0.07}\) & \(4.77\pm{0.05}\) & \(9.04\pm{0.11}\) \\ \cline{2-5}
            & 4 & \(8.25\pm{0.03}\) & \(6.09\pm{0.04}\) & \(9.92\pm{0.00}\) \\ \cline{2-5}
            & 5 & \(11.96\pm{0.09}\) & \(7.25\pm{0.02}\) & \(12.09\pm{0.08}\) \\ \cline{2-5}
            & 6 & \(12.23\pm{0.09}\) & \(8.06\pm{0.02}\) & \(12.84\pm{0.04}\) \\ \cline{2-5}
            & 7 & \(13.77\pm{0.04}\) & \(9.78\pm{0.03}\) & \(13.59\pm{0.05}\) \\ \cline{2-5}
            & 8 & \(14.93\pm{0.04}\) & \(10.70\pm{0.03}\) & \(14.98\pm{0.08}\) \\ \hline
        \end{tabular}
        \end{adjustbox}
    \end{subfigure}
    \hfill
    \begin{subfigure}{0.5\textwidth}
        \includegraphics[width=\textwidth]{bucle3=03-large}
    \end{subfigure}
    \caption{\underline{Tamaño largo}: Tiempos de ejecución vs nº de hilos}
    \label{bucle3=03-large}
\end{figure}

\par Estos dos bucles anidados también podría ponerse el pragma \texttt{\#pragma omp parallel for collapse(2) reduction (max:difference)}
pero comprobando los tiempos se puede ver que empeora el tiempo de ejecución.


\par El primer y tercer bucle no se pueden vectorizar de forma automática porque tiene dependencias. No son inmediatamente
vectorizables, tampoco se puede reorganizar el código para evitar las dependencias ni crear instrucciones parciales para mitigar la
dependencia. No he encontrado ninguna transformación manual para poder vectorizarlos. Separar cada bucle en dos sería
ineficiente y el compilador descarta la vectorización de la instrucción que ya no tendría dependencia por tanto este cambio no
sería una solución. A lo mejor se puede programar usando los \texttt{instrinsics} vectoriales, que son un conjunto de funciones
disponibles para lenguajes de alto nivel que corresponden directamente a instrucciones vectoriales o hay alguna librería
optimizada que ya esté vectorizada, porque la idea de programar en lenguaje ensamblador hoy en día casi nunca se realiza.