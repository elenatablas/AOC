%%%%%%%%%%%%%%%%%%%%%% Procesadores CMP: usando la vectorización y la paralelización - EJERCICIO 1 %%%%%%%%%%%%%%%%%%%%%%
\section{Ejercicio 1}
\subsection{Enunciado}
\begin{ejer}
    \textbf{1.-} En el subdirectorio \texttt{vectorización} hay un pequeño programa que tiene 6 bucles.
    \begin{itemize}
        \item Para cada uno de los bucles del programa, identifica las dependencias entre iteraciones que tiene
    y explica cómo afectarían a una posible vectorización automática.
        \item Utiliza el \texttt{Makefile} incluido para compilar el programa con el compilador \texttt{ICC} y generar
    el informe de optimización. Para cada bucle mencionado en el informe, explica por qué ha sido vectorizado o por qué no 
    se ha podido vectorizar. En especial, para los bucles en los que se hubieran identificado dependencias entre iteraciones
     en el punto anterior:
        \begin{itemize}
            \item Explica en caso de que haya sido vectorizado qué transformaciones ha aplicado automáticamente el compilador.
            \item Y en caso de que no haya sido vectorizado, indica si crees que existe alguna transformación manual posible para
            que se pudiera vectorizar.
        \end{itemize}
    \end{itemize}
\end{ejer}
\subsection{Desarrollo}
